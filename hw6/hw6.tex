\documentclass{article}
\usepackage[utf8]{inputenc}
\usepackage{amsmath}
\usepackage{amssymb}
\usepackage{amsfonts}
\usepackage{amssymb}
\usepackage{minted}
\usepackage{graphicx}
\usepackage{algorithm}
\usepackage{algpseudocode}
\graphicspath{ {img/} }
\usepackage{titlesec}
\usepackage[a4paper,margin=1in,footskip=0.25in]{geometry}
\usepackage{fancyhdr}
\pagestyle{fancy}
%basic page layout

%draw finite state machine
\usepackage{tikz}
\usetikzlibrary{arrows,automata}
\newcommand{\hwnumber}{6}
\newcommand{\Lcvy}{\mathcal{L}}
%header and footer settings
\lhead{Algorithms and Data Structure \hwnumber}
\chead{Yiping Deng}
\rhead{\today}

\titlelabel{\thetitle\enspace}

\begin{document}
\title{Algorithms and Data Structure \hwnumber}
\author{Yiping Deng}
\maketitle
\thispagestyle{fancy}
\section*{Problem 1}
\subsection*{a)}
We can see the python code here and explain it.
\inputminted{Python}{counting_sort.py}
Basically, in counting sort, we will count the appearance of
each entries in the array, aggregate the counts in a ascending
order, and then using such information to put every entries
in the right position.
We starts by running the script, we have
\begin{verbatim}
counting of 9 is incremented
counting of 1 is incremented
counting of 6 is incremented
counting of 7 is incremented
counting of 6 is incremented
counting of 2 is incremented
counting of 1 is incremented
countings: [0, 2, 1, 0, 0, 0, 2, 1, 0, 1]
now we aggregate the countings
countings: [0, 2, 3, 3, 3, 3, 5, 6, 6, 7]
restore the result
put 9 at index 6
put 1 at index 1
put 6 at index 4
put 7 at index 5
put 6 at index 3
put 2 at index 2
put 1 at index 0
[1, 1, 2, 6, 6, 7, 9]
\end{verbatim}
\subsection*{b)}
Similarly, we have a python script that executes the
bucket sort and print the intermediate steps.
It first put elements in the array into $n$ equidistance
buckets, or arrays, and then do insertion sort on every
buckets.
\inputminted{Python}{bucket_sort.py}
Let's run this script and see this algorithm is executed.
\begin{verbatim}
number of bucket: 7
[[], [], [], [], [], [], []]
build empty buckets
start to put in buckets
find bucket for 0.9
put in bucket 6
now in buckets we have [[], [], [], [], [], [], [0.9]]
find bucket for 0.1
put in bucket 0
now in buckets we have [[0.1], [], [], [], [], [], [0.9]]
find bucket for 0.6
put in bucket 4
now in buckets we have [[0.1], [], [], [], [0.6], [], [0.9]]
find bucket for 0.7
put in bucket 4
now in buckets we have [[0.1], [], [], [], [0.6, 0.7], [], [0.9]]
find bucket for 0.6
put in bucket 4
now in buckets we have [[0.1], [], [], [], [0.6, 0.7, 0.6], [], [0.9]]
find bucket for 0.2
put in bucket 1
now in buckets we have [[0.1], [0.2], [], [], [0.6, 0.7, 0.6], [], [0.9]]
find bucket for 0.1
put in bucket 0
now in buckets we have [[0.1, 0.1], [0.2], [], [], [0.6, 0.7, 0.6], [], [0.9]]
calling the insertion sort on [0.1, 0.1]
building results by adding [0.1, 0.1]
calling the insertion sort on [0.2]
building results by adding [0.2]
calling the insertion sort on []
building results by adding []
calling the insertion sort on []
building results by adding []
calling the insertion sort on [0.6, 0.7, 0.6]
building results by adding [0.6, 0.6, 0.7]
calling the insertion sort on []
building results by adding []
calling the insertion sort on [0.9]
building results by adding [0.9]
[0.1, 0.1, 0.2, 0.6, 0.6, 0.7, 0.9]
\end{verbatim}
\subsection*{c)}
We basically implement the first half of bucket sort. We build $k$ buckets and
put them in accordingly. Without aggregating, to calculate the number of integer in
inteval $[a, b]$, we just add up numbers from bucket $a$ to bucket $b$, and it will give
you the result.
\subsection*{d)}
the concrete implementation is here
\inputminted{Python}{word_sort.py}
\subsection*{e)}
Let's implement a variant of bubblesort
\begin{algorithm}
    \caption{Distance}
    \label{Distance}
    \begin{algorithmic}[1]
        \Procedure{Distance}{$p$}
        \Comment{Input a point $p$}
        \State $d_{p} \gets \sqrt{p_1^2 + p_2^2}$ \\
        \Return $d_{p}$
        \EndProcedure
    \end{algorithmic}
\end{algorithm}
\begin{algorithm}
    \caption{GreaterThan}
    \label{GreaterThan}
    \begin{algorithmic}[1]
        \Procedure{TessThan}{$p, q$}
        \Comment{Compare two points $p, q$}
        \State $cmp \gets Distance(p) > Distance(q)$ \\
        \Return $cmp$
        \EndProcedure
    \end{algorithmic}
\end{algorithm}

\begin{algorithm}
    \caption{Bubble Sort}
    \label{bubblesort}
    \begin{algorithmic}[1]
        \Procedure{BubbleSort}{$A$}
        \Comment{Input a array $A$}
        \State $swapped \gets true$
        \Comment{Make sure it will execute at least once}
        \While {swapped}
        \Comment{Keep executing until it is sorted}
        \State $swapped \gets false$
        \For {$i$ from 1 to A.size - 1}
        \If{$GreaterThan(A[i], A[i + 1])$}
        \State $swapped \gets true$
        \State $swap(A[i], A[i + 1])$
        \Comment{Swap two misplaced elements in the array}
        \EndIf
        \EndFor
        \EndWhile
        \EndProcedure
    \end{algorithmic}
\end{algorithm}
\section*{Problem 2}
\subsection*{a)}
This is the variant of radix sort here.
\inputminted{Python}{radix_sort.py}
\subsection*{b)}
Time complexity is straight forward. Assuming a uniform
distribution on the input, every step will divide the input
into $base$ cases, and every depth you have $n$ elements to put in and out
of buckets in total, with
a tree depth of number of digits, denoted as $k$. Thus, the
time complexity is $O(k * n) = O(n)$ \\
Space complexity will be also $O(k * n) = O(n)$. Every recursive call
you will put $n$ elements in and out of the buckets, thus
it will have a space complexity of $O(k * n) = O(n)$
\subsection*{c)}
You simply take $base = n$ as basis for Radix sort. Thus, it will 
be at most $3$ as the depth of recursive call(at most 3 digit), and every
recursive call will take $n$ running time, leading to total
$O(3n) = O(n)$
\end{document}
