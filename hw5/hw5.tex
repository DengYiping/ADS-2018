\documentclass{article}
\usepackage[utf8]{inputenc}
\usepackage{amsmath}
\usepackage{amssymb}
\usepackage{amsfonts}
\usepackage{amssymb}
\usepackage{minted}
\usepackage{graphicx}
\usepackage{algorithm}
\usepackage{algorithmic}
\usepackage{mathtools}
\providecommand{\ceilfunc}[1]{\left \lceil #1 \right \rceil }
\providecommand{\floorfunc}[1]{\left \lfloor #1 \right \rfloor }
\graphicspath{ {img/} }
\usepackage{titlesec}
\usepackage[a4paper,margin=1in,footskip=0.25in]{geometry}
\usepackage{fancyhdr}
\pagestyle{fancy}
%basic page layout

%draw finite state machine
\usepackage{tikz}
\usetikzlibrary{arrows,automata}
\newcommand{\hwnumber}{5}
\newcommand{\Lcvy}{\mathcal{L}}
%header and footer settings
\lhead{Algorithms and Data Structure \hwnumber}
\chead{Yiping Deng}
\rhead{\today}

\titlelabel{\thetitle\enspace}

\begin{document}
\title{Algorithms and Data Structure \hwnumber}
\author{Yiping Deng}
\maketitle
\thispagestyle{fancy}
\section*{Problem 1}
\subsection*{a)}
File 1: QuickSortVariant.scala
\inputminted{Scala}{QuickSort/src/main/scala/QuickSortVariant.scala}
Test case \\
File 2: QuickSortTest.scala
\inputminted{Scala}{QuickSort/src/test/scala/QuickSortTest.scala}
\subsection*{b)}
\textbf{Best Case:}
Best case will lead to equal partition in all recursive calls.
Every recursions will make 3 recursive calls on a $n/3$ data
set.
Partition iterates through the element once,
so it's complexity is $O(n)$.
Thus we can write the recurrences, and solve it. \\
\begin{align*}
    T(n) &= 3 T(n) + O(n) \\
    T(n) &= O(n \log(n))
\end{align*}
Hence, the best case is $O(n \log(n))$ \\
\textbf{Worst Case:}
Worst case will lead to partition that place the two
pivots at the beginning and at the end.
The first partition contains $0$ element, and the second
partition contains $n - 2$ elements, and the third
partition contains $0$ elements.
Essentially, it just put the two pivot at the right position
on every recursive call, nothing else.
Partition iterates through the element once,
so it's complexity is $O(n)$.
Thus, we can write the recurrence as following and solve it
\begin{align*}
    T(n) &= T(n - 2) + O(n) \\
    T(n) &= O(n^2)
\end{align*}
Hence, the worst case is $O(n^2)$.
\subsection*{c)}
Let's modify our partition a bit, and change nothing else.

File 1: QuickSortVariantRandom.scala
\inputminted{Scala}{QuickSort/src/main/scala/QuickSortVariantRandom.scala}

\section*{Problem 2}
\subsection*{a)}
\textbf{Proof:} We start by simpify the LHS
\begin{align*}
    LHS &\leq \sum_{k = 1}^{n - 1} k \lg(k) \leq
    \sum_{k = 1}^{\floorfunc{n / 2}} k \lg(k) +
    \sum_{k = \floorfunc{n / 2} + 1}^{n - 1} k \lg(k) \\
    &\leq lg(\floorfunc{n / 2}) \sum_{k = 1}^{\floorfunc{n / 2}} k  +
    \lg(n) \sum_{k = \floorfunc{n / 2} + 1}^{n - 1} k \lg(k) \\
    &\leq (\lg(n) - 1) \sum_{k = 1}^{\floorfunc{n / 2}} k  +
    \lg(n) \sum_{k = \floorfunc{n / 2} + 1}^{n - 1} k \lg(k) \\
    &\leq \lg(n) \sum_{k = 1}^{n - 1} k -  \sum_{k = 1}^{\floorfunc{n / 2}} k
\end{align*}
\textbf{Lemma:} $\sum_{k = 1}^{n} = \frac{n(n + 1)}{2} $ \\
Using induction: \\
Basis: $ n = 1 $
\begin{align*}
    \sum_{k = 1}^{n} k  &= 1 = \frac{1 \cdot (1 + 1)}{2} = 1 \\
\end{align*}
Inductive step: Assume it holds for $n$
\begin{align*}
    \sum_{k = 1}^{n + 1} k &= \sum_{k = 1}^{n} k + n  + 1\\
    &= \frac{n(n+1)}{2} + \frac{2n + 2}{2} \\
    &= \frac{n^2 + 3n + 2}{2} \\
    &= \frac{(n + 1)(n + 2)}{2}
\end{align*}
Using this to simplify the formula, we have
\begin{align*}
    LHS &\leq \lg(n) \frac{n(n - 1)}{2} - \frac{n/2 \cdot (n/2 + 1)}{2} \\
    &= \lg(n) \frac{n^2 / 2}{2} - \frac{n}{2} \lg(n)
    -\frac{n^2}{8} - \frac{n}{4} \lg(n) \\
    &\leq \frac{1}{2} n^2 \lg(n) - \frac{1}{8} n^2
\end{align*}
\subsection*{b)}
\textbf{Assumption:} Uniform distribution of permutation. \\
\textbf{Proof:}
define random variable
\begin{align*}
    X_k :=
    \begin{cases}
        1 & \text{if partition at index k}\\
        0 & \text{if not}\\
    \end{cases}
\end{align*}
Due to uniform distribution, we have
$$ E(X_k) = 1/n $$
Thus,
\begin{align*}
    T(n) :=
    \begin{cases}
        T(0) + T(n - 1) + \Theta(n) & \text{if partition at index 0}\\
        T(1) + T(n - 2) + \Theta(n) & \text{if partition at index 1}\\
        &...\\
        &...\\
        T(n - 1) + T(0) + \Theta(n) & \text{if partition at index } n - 1\\
    \end{cases}
\end{align*}
Taking expectation on both side and use linearity of expectation. \\
Also merge two similar sum. Merge the case $k = 0, k = 1$
into the term $\Theta(n)$
\begin{align*}
    E(T(n)) &=
    E(\sum_{k = 0}^{n - 1} X_k (T(k) + T(n - k - 1)) + ) \\
    &= 1/n \sum_{k = 0}^{n - 1} E(T(k)) +
    1/n  \sum_{k = 0}^{n - 1} E(T(n - k - 1))
    + 1/n  \sum_{k = 0}^{n - 1} \Theta(n) \\
    &= \frac{2}{n} \sum_{k = 1}^{n - 1} E(T(k)) + \Theta(n)
\end{align*}
Base case: $k = 2$, it is quite obvious since $c$ can be arbitrary big. \\
Inductive case:(strong induction) \\
Assume the argument holds for any $n \leq m -1 $
\begin{align*}
    E(T(m)) &\leq \frac{2}{n} \sum_{k = 2}{m - 1} c \cdot k \lg(k) + \Theta(n)\\
    &\leq \frac{2c}{n} (\frac{1}{2} n^2 \lg(n) - \frac{1}{8} n^2)\\
    &\leq c n \lg(n) - \frac{c n}{4} + \Theta(n) \\
    &\leq c n \lg(n) \text{ for arbitrary c}
\end{align*}
\section*{Problem 3:}
\textbf{Part 1:} Upper bound.
\begin{align*}
    n! &\leq n^n \implies \\
    \lg(n!) &\leq \lg(n^n) = n \lg(n) \\
\end{align*}
\textbf{Part 2:} Lower bound.
\begin{align*}
    (n!)^2 &\geq n^n \implies \\
    \lg((n!)^2) &= 2 \lg(n!) \geq n \lg(n) \implies \\
    \lg(n!) &\geq \frac{1}{2} n \lg(n)
\end{align*}
This directly implies that
$$ \lg(n!) = \Theta(n \lg(n)) $$
\end{document}
