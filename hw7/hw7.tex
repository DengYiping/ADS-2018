\documentclass{article}
\usepackage[utf8]{inputenc}
\usepackage{amsmath}
\usepackage{amssymb}
\usepackage{amsfonts}
\usepackage{amssymb}
\usepackage{minted}
\usepackage{graphicx}
\usepackage{algorithm}
\usepackage{algorithmic}
\graphicspath{ {img/} }
\usepackage{titlesec}
\usepackage[a4paper,margin=1in,footskip=0.25in]{geometry}
\usepackage{fancyhdr}
\pagestyle{fancy}
%basic page layout

%draw finite state machine
\usepackage{tikz}
\usetikzlibrary{arrows,automata}
\newcommand{\hwnumber}{7}
\newcommand{\Lcvy}{\mathcal{L}}
%header and footer settings
\lhead{Algorithms and Data Structure \hwnumber}
\chead{Yiping Deng}
\rhead{\today}

\titlelabel{\thetitle\enspace}

\begin{document}
\title{Algorithms and Data Structure \hwnumber}
\author{Yiping Deng}
\maketitle
\thispagestyle{fancy}
\section*{Problem 1}
\subsection*{a)}
The following is the concrete implementation in Python. Python is by-default generic, since it is not statically typed language. \\
\textbf{Note:} The python code includes a small unit test. You can simply run
\begin{minted}{bash}
    $ python -m doctest -v stack.py
\end{minted}
\inputminted{Python}{stack.py}
\subsection*{b)}
We can implement a queue via two stacks. The first stack store the input, or freshly enqueued elements. The second stack store the output, or the elements
that is ready is dequeue.
When we try to dequeue, and if the second stack is empty, dump the first stack into the second one(poping from stack 1 and push into stack 2).
The implementation in Python is also included here:

\textbf{Note:} The python code includes a small unit test. You can simply run
\begin{minted}{bash}
    $ python -m doctest -v queue.py
\end{minted}
\inputminted{Python}{queue.py}
\section*{Problem 2}
\subsection*{a)}
\end{document}
